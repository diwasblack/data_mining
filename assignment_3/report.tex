\documentclass{article}
\usepackage[utf8]{inputenc}
\usepackage{graphicx}
\usepackage{listings}
\usepackage{amsmath} 

\title{Assignment for CS 641 \\ (Data Mining)}
\author{Diwas Sharma}
\date{\today}
\begin{document}

\maketitle
\newpage

\section{Linear Discriminant Analysis}

\subsection{Linear Discriminant Analysis}
\subsubsection{Decision boundary}
For any point $X = \begin{bmatrix}x \\ y\end{bmatrix}$, the general decision boundary of LDA for the given problem is shown in
equation \ref{eq:lda_equation}.

\begin{equation}
    \label{eq:lda_equation}
    (X-\mu_{1})^{T}\Sigma^{-1}(X - \mu_{1}) - (X-\mu_{2})^{T}\Sigma^{-1}(X - \mu_{2}) = 0
\end{equation}

where, $\mu{1}$ is the average of class 1,
$\mu{2}$ is the average of class 2, and
$\Sigma$ is the covariance matrix of the data

The equation can be simplified to obtain equation \ref{eq:lda_equation_simple} which is a linear
equation in $X$.

\begin{equation}
    \label{eq:lda_equation_simple}
    (\Sigma^{-1}\mu_{1} -  \Sigma^{-1}\mu_{2}) X + \frac{1}{2}(-\mu^{T}_{1}\Sigma^{-1}\mu_{1} + \mu^{T}_{2}\Sigma^{-1}\mu_{2}) = 0
\end{equation}

If we plot the decision boundary, we would get results similar to one
shown in figure \ref{fig:lda_decision_boundary}.


\begin{figure}[!ht]
  \includegraphics[width=\textwidth,height=0.4\textheight,keepaspectratio]{lda.png}
  \caption{Decision boundary for LDA}
  \label{fig:lda_decision_boundary}
\end{figure}

\subsection{Quadratic Discriminant Analysis}
\subsubsection{Decision boundary}
Similarly, the general decision boundary of QDA for the given problem is shown in equation \ref{eq:qda_equation}, which is a quadratic equation in X.

\begin{equation}
    \label{eq:qda_equation}
    (X-\mu_{1})^{T}\Sigma^{-1}_{1}(X - \mu_{1}) - (X-\mu_{2})^{T}\Sigma^{-1}_{2}(X - \mu_{2})
    + \frac{1}{2}(-\ln{\begin{vmatrix}\Sigma_{1}\end{vmatrix}} + \ln{\begin{vmatrix}\Sigma_{2}\end{vmatrix}})= 0
\end{equation}

where, $\Sigma_{1}$ is the covariance of data that belongs to class 1, and $\Sigma_{2}$ is the covariance of data that belongs to class 2.

The plot of decision boundary obtain is shown in figure \ref{fig:qda_decision_boundary}.

\begin{figure}[!ht]
  \includegraphics[width=\textwidth,height=0.4\textheight,keepaspectratio]{qda.png}
  \caption{Decision boundary for QDA}
  \label{fig:qda_decision_boundary}
\end{figure}

\end{document}
