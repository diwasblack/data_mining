\documentclass{article}
\usepackage[utf8]{inputenc}
\usepackage{graphicx}
\usepackage{listings}

\title{Assignment for CS 641 \\ (Data Mining)}
\author{Diwas Sharma}
\date{\today}
\begin{document}

\maketitle
\newpage

\section{Linear Discriminant Analysis}

\subsection{Linear Discriminant Analysis}
\subsubsection{Decision boundary}
For any two dimensionaly point $(x, y)$, the decision boundary of LDA is
equation \ref{eq:lda_equation} whose plot is shown in figure \ref{fig:lda_decision_boundary}

\begin{equation}
    \label{eq:lda_equation}
    0.798 x + 2.82 y - 5.556 = 0
\end{equation}

\begin{figure}[!ht]
  \includegraphics[width=\textwidth,height=0.4\textheight,keepaspectratio]{lda.png}
  \caption{Decision boundary for LDA}
  \label{fig:lda_decision_boundary}
\end{figure}

\subsection{Quadratic Discriminant Analysis}
\subsubsection{Decision boundary}
Similarly, the decision boundary of QDA is equation \ref{eq:qda_equation}
whose plot is shown in figure \ref{fig:qda_decision_boundary}
\begin{equation}
    \label{eq:qda_equation}
    1.681 x^{2} - 6.0181 x + 2.092 y^{2} + 5.497y + 5.722xy - 18.873
\end{equation}

\begin{figure}[!ht]
  \includegraphics[width=\textwidth,height=0.4\textheight,keepaspectratio]{qda.png}
  \caption{Decision boundary for QDA}
  \label{fig:qda_decision_boundary}
\end{figure}

\section{Source Code}
\begin{lstinputlisting}[language=python]{gaussian_classifier.py}
\end{lstinputlisting}

\end{document}
