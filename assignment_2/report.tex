\documentclass{article}
\usepackage[utf8]{inputenc}
\usepackage{graphicx}
\usepackage{listings}

\title{Assignment for CS 641 \\ (Data Mining)}
\author{Diwas Sharma}
\date{\today}

\begin{document}

\maketitle
\newpage

\section{}
\subsection{Source Code}
\begin{lstinputlisting}[language=python]{knn.py}
\end{lstinputlisting}

\subsection{Runtime performance of predict function}

For K = 25, \newline
19.3 ms $\pm$ 719 $\mu$s per loop (mean $\pm$ std dev of 7 runs, 10 loops each)\newline
For K = 5, \newline
18.6 ms $\pm$ 473 $\mu$s per loop (mean $\pm$ std dev of 7 runs, 10 loops each)\newline
For K = 1, \newline
19.6 ms $\pm$ 3.26 ms per loop (mean $\pm$ std dev of 7 runs, 10 loops each)


\subsection{Plot of Accuracy against K}
\begin{figure}[!ht]
  \includegraphics[width=\textwidth,height=0.4\textheight,keepaspectratio]{validation_accuracy.png}
  \caption{Plot of accuracy against K}
  \label{fig:accuracy_plot}
\end{figure}
The plot of accuracy against K is shown in figure \ref{fig:accuracy_plot}

\subsection{Best K for the highest accuracy}
One of the value that has the highest accuracy is K=1

\subsection{Accuracy}
The accuracy obtained for the best K having the highest accuracy cannot be
used to report the accuracy of the algorithm to the user because the hyperparameter
K might overfit the test data which will result in better training performance
but a poor generalization performance for new data.

\end{document}
