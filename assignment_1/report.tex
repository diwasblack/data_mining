\documentclass{article}
\usepackage[utf8]{inputenc}
\usepackage{listings}

\title{Assignment for CS 641 \\ (Data Mining)}
\author{Diwas Sharma}
\date{\today}

\begin{document}

\maketitle
\newpage

\section{Problem 1}
The outcome of 6-sided dice can be modelled using the multinomial
distribution. Let $\theta_{i}$ be the probability that side $i$ shows up,
and $n_{i}$ the number of times the side ${i}$ appears in the experiment. Then we
have,

\begin{equation}
\label{eq:theta_sum}
\sum_{1}^{6}\theta_{i} = 1
\end{equation}

\begin{equation}
\label{eq:n_sum}
\sum_{1}^{6}n_{i} = n
\end{equation}
where $n$ is the total number of experiments.

Then, the likelihood of the data given $\theta$s, can be calculated as

\begin{equation}
l(n|\theta) = \theta_{1}^{n_1}\theta_{2}^{n_2}\theta_{3}^{n_3}\theta_{4}^{n_4}\theta_{5}^{n_5}\theta_{6}^{n_6}
\end{equation}

After taking log of likelihood function we get,
\begin{equation}
\label{eq:objective}
L(\theta_{1}\theta_{2}..\theta_{6})=\sum_{1}^{6}n_i.log(\theta_{i})
\end{equation}

Then The goal is to maximize equation~\ref{eq:objective} subjected to
constraint given in equation~\ref{eq:theta_sum}. This can be done by 
adding lagrange multipliers into the equation~\ref{eq:objective}
\begin{equation}
F(\theta_{1}..\theta_{6},\lambda) = L(\theta_{1}\theta_{2}..\theta_{6}) + \lambda.(\sum_{1}^6\theta_{i} - 1)
\end{equation}

We can solve for $\theta$s by setting the gradient of the function to zero.

\begin{equation}
\nabla F(\theta_{1}..\theta_{6},\lambda) = 0
\end{equation}

By solving above equation we get,

\begin{equation}
\label{eq:theta_lambda}
\theta_{i} = -\frac{n_{i}}{\lambda}
\end{equation}

Substituting equation~\ref{eq:theta_lambda} into equation~\ref{eq:theta_sum} and
using equation~\ref{eq:n_sum}, we get $\lambda=-n$. So the equation~\ref{eq:theta_lambda}
becomes

\begin{equation}
\theta_{i} = \frac{n_{i}}{n}
\end{equation}
which is the required probability for the face $i$.

\section{Problem 2}

\begin{lstinputlisting}[language=python]{pageranking.py}
\end{lstinputlisting}

\end{document}
